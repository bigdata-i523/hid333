\documentclass[sigconf]{acmart}

\usepackage{graphicx}
\usepackage{hyperref}
\usepackage{todonotes}

\usepackage{endfloat}
\renewcommand{\efloatseparator}{\mbox{}} % no new page between figures

\usepackage{booktabs} % For formal tables

\settopmatter{printacmref=false} % Removes citation information below abstract
\renewcommand\footnotetextcopyrightpermission[1]{} % removes footnote with conference information in first column
\pagestyle{plain} % removes running headers

\newcommand{\TODO}[1]{\todo[inline]{#1}}

\begin{document}
\title{Big Data and Artificial Intelligence solutions for In Home, Community and Territory Security}


\author{Anil Ravi}
\orcid{HID333}
\affiliation{%
  \institution{Indiana University}
  \streetaddress{711 N Park Ave}
  \city{Bloomington} 
  \state{Indiana} 
  \postcode{47408}
}
\email{anilravi@iu.edu}

\author{Ashok Reddy Singam}
\orcid{HID337}
\affiliation{%
  \institution{Indiana University}
  \streetaddress{711 N Park Ave}
  \city{Bloomington} 
  \state{Indiana} 
  \postcode{47408}
}
\email{asingam@iu.edu}

\begin{abstract}
Smart Security systems equipped with Video and Audio sensors produce huge amounts unstructured data continuously. 
This paper talks about high level architecture of an intelligent security system with video surveillance,  audio monitoring/recording, video and Audio analytics, alerting homeowners/authorities/agencies as needed.

\end{abstract}

\keywords{Big Data, AI}


\maketitle

\section{Introduction}

Having an intelligent ear-and-eye monitoring at the home to constantly observe the surroundings both inside and outside can protect the house \TODO{\&} personnel much more safer way. By extending this capability to the neighborhood and city through collaboration would create safe cities across the world. 

\section{Smart Security Systems}

Typically, the body of a paper is organized into a hierarchical
structure, with numbered or unnumbered headings for sections,
subsections, sub-subsections, and even smaller sections.  The command
\texttt{{\char'134}section} that precedes this paragraph is part of
such a hierarchy. \LaTeX\ handles the
numbering and placement of these headings for you, when you use the
appropriate heading commands around the titles of the headings.  If
you want a sub-subsection or smaller part to be unnumbered in your
output, simply append an asterisk to the command name.  Examples of
both numbered and unnumbered headings will appear throughout the
balance of this sample document.

Because the entire article is contained in the \textbf{document}
environment, you can indicate the start of a new paragraph with a
blank line in your input file; that is why this sentence forms a
separate paragraph.

\subsection{Type Changes and {\itshape Special} Characters}

We have already seen several typeface changes in this sample.  You can
indicate italicized words or phrases in your text with the command
\texttt{{\char'134}textit}; emboldening with the command
\texttt{{\char'134}textbf} and typewriter-style (for instance, for
computer code) with \texttt{{\char'134}texttt}.  But remember, you do
not have to indicate typestyle changes when such changes are part of
the \textit{structural} elements of your article; for instance, the
heading of this subsection will be in a sans serif\footnote{Another
  footnote here.  Let's make this a rather long one to see how it
  looks. Footnotes must be avoided.} typeface, but that is handled by
the document class file.  Take care with the use of the curly braces
in typeface changes; they mark the beginning and end of the text that
is to be in the different typeface.

You can use whatever symbols, accented characters, or non-English
characters you need anywhere in your document; you can find a complete
list of what is available in the \textit{\LaTeX\ User's Guide}
\cite{Lamport:LaTeX}.

\subsection{Math Equations}

You may want to display math equations in three distinct styles:
inline, numbered or non-numbered display.  Each of
the three are discussed in the next sections.

\subsubsection{Inline (In-text) Equations}

A formula that appears in the running text is called an
inline or in-text formula.  It is produced by the
\textbf{math} environment, which can be
invoked with the usual \texttt{{\char'134}begin\,\ldots{\char'134}end}
construction or with the short form \texttt{\$\,\ldots\$}. You
can use any of the symbols and structures,
from $\alpha$ to $\omega$, available in
\LaTeX~\cite{Lamport:LaTeX}; this section will simply show a
few examples of in-text equations in context. Notice how
this equation:

\begin{math}
  \lim_{n\rightarrow \infty}x=0
\end{math},

set here in in-line math style, looks slightly different when
set in display style.  (See next section).

\subsubsection{Display Equations}

A numbered display equation---one set off by vertical space from the
text and centered horizontally---is produced by the \textbf{equation}
environment. An unnumbered display equation is produced by the
\textbf{displaymath} environment.

Again, in either environment, you can use any of the symbols
and structures available in \LaTeX\@; this section will just
give a couple of examples of display equations in context.
First, consider the equation, shown as an inline equation above:

\subsection{Citations}

Citations to articles~\cite{bowman:reasoning, clark:pct, braams:babel,
  herlihy:methodology}, conference proceedings~\cite{clark:pct} or
maybe books \cite{Lamport:LaTeX, salas:calculus} listed in the
Bibliography section of your article will occur throughout the text of
your article.  You should use BibTeX to automatically produce this
bibliography; you simply need to insert one of several citation
commands with a key of the item cited in the proper location in the
\texttt{.tex} file~\cite{Lamport:LaTeX}.  The key is a short reference
you invent to uniquely identify each work; in this sample document,
the key is the first author's surname and a word from the title.  This
identifying key is included with each item in the \texttt{.bib} file
for your article.

The details of the construction of the \texttt{.bib} file are beyond
the scope of this sample document, but more information can be found
in the \textit{Author's Guide}, and exhaustive details in the
\textit{\LaTeX\ User's Guide} by Lamport~\shortcite{Lamport:LaTeX}.

This article shows only the plainest form of the citation command,
using \texttt{{\char'134}cite}.

Some examples.  A paginated journal article \cite{Abril07}, an
enumerated journal article \cite{Cohen07}, a reference to an entire
issue \cite{JCohen96}, a monograph (whole book) \cite{Kosiur01}, a
monograph/whole book in a series (see 2a in spec. document)
\cite{Harel79}, a divisible-book such as an anthology or compilation
\cite{Editor00} followed by the same example, however we only output
the series if the volume number is given \cite{Editor00a} (so
Editor00a's series should NOT be present since it has no vol. no.), a
chapter in a divisible book \cite{Spector90}, a chapter in a divisible
book in a series \cite{Douglass98}, a multi-volume work as book
\cite{Knuth97}, an article in a proceedings (of a conference,
symposium, workshop for example) (paginated proceedings article)
\cite{Andler79}, a proceedings article with all possible elements
\cite{Smith10}, an example of an enumerated proceedings article
\cite{VanGundy07}, an informally published work \cite{Harel78}, a
doctoral dissertation \cite{Clarkson85}, a master's thesis:
\cite{anisi03}, an online document / world wide web resource
\cite{Thornburg01, Ablamowicz07, Poker06}, a video game (Case 1)
\cite{Obama08} and (Case 2) \cite{Novak03} and \cite{Lee05} and (Case
3) a patent \cite{JoeScientist001}, work accepted for publication
\cite{rous08}, 'YYYYb'-test for prolific author \cite{SaeediMEJ10} and
\cite{SaeediJETC10}. Other cites might contain 'duplicate' DOI and
URLs (some SIAM articles) \cite{Kirschmer:2010:AEI:1958016.1958018}.
Boris / Barbara Beeton: multi-volume works as books \cite{MR781536}
and \cite{MR781537}.

A couple of citations with DOIs: \cite{2004:ITE:1009386.1010128,
  Kirschmer:2010:AEI:1958016.1958018}.

Online citations: \cite{TUGInstmem, Thornburg01, CTANacmart}.  

We use jabref to manage all citations. A paper without managing a bib
file will be returned without review. in the bibtex file all urls are
added to rfernces with the {\it url} filed. They are not to be
included in the {\it howpublished} or {\it note} field. 

\section{Conclusions}

This paragraph will end the body of this sample document.  Remember
that you might still have Acknowledgments or Appendices; brief samples
of these follow.  There is still the Bibliography to deal with; and we
will make a disclaimer about that here: with the exception of the
reference to the \LaTeX\ book, the citations in this paper are to
articles which have nothing to do with the present subject and are
used as examples only.

\begin{acks}

  The authors would like to thank Dr. Yuhua Li for providing the
  matlab code of the \textit{BEPS} method.

  The authors would also like to thank the anonymous referees for
  their valuable comments and helpful suggestions. The work is
  supported by the \grantsponsor{GS501100001809}{National Natural
    Science Foundation of
    China}{http://dx.doi.org/10.13039/501100001809} under Grant
  No.:~\grantnum{GS501100001809}{61273304}
  and~\grantnum[http://www.nnsf.cn/youngscientsts]{GS501100001809}{Young
    Scientsts' Support Program}.

\end{acks}

\bibliographystyle{ACM-Reference-Format}
\bibliography{report} 

\end{document}
